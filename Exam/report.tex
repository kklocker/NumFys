\documentclass[11pt,a4paper,twocolumn]{article}
\usepackage[utf8]{inputenc}
\usepackage[T1]{fontenc}
\usepackage{amsmath}
\usepackage{amsfonts}
\usepackage{amssymb}
\usepackage{graphicx}
\usepackage{hyperref}

\hypersetup{
	colorlinks=true,
	linkcolor=blue,
	filecolor=magenta,      
	urlcolor=cyan,
}

\author{Karl Kristian Ladegård Lockert}
\title{TFY4235: Exam}
\begin{document}
\maketitle
\begin{abstract}

Exam solutions for TFY4235: Computational physics spring 2020.

\end{abstract}
		
\section{Assignment 2}
The final report for Assignment 2: Biased Brownian motion are attached together with code for the project. The core functionality of the program is written in Python, with heavy utilisation of compiled packages, as well as functionality for compiling python code. Most plots and development have been done in Jupyter Notebooks, which can be found in the Notebooks folder. If you do not have a Notebook viewer available, the very same files can be found and viewed in my open \href{https://github.com/kklocker/NumFys}{GitHub repository}. These files are, however, much more experimental and less documented, as they are created for very specific tasks or testing. 
\end{document}

