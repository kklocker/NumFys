\documentclass[5p]{elsarticle}			% 5p gir 2 kolonner pr side. 1p gir 1 kolonne pr side.
\journal {Veileder}
\usepackage[T1]{fontenc} 				% Vise norske tegn.
%\usepackage[norsk]{babel}				% Tilpasning til norsk.
\usepackage{graphicx}       				% For å inkludere figurer.
\usepackage{amsmath,amssymb} 			% Ekstra matematikkfunksjoner.
%\usepackage{siunitx}					% Må inkluderes for blant annet å få tilgang til kommandoen \SI (korrekte måltall med enheter)
%	\sisetup{exponent-product = \cdot}      	% Prikk som multiplikasjonstegn (i steden for kryss).
% 	\sisetup{output-decimal-marker  =  {,}} 	% Komma som desimalskilletegn (i steden for punktum).
% 	\sisetup{separate-uncertainty = true}   	% Pluss-minus-form på usikkerhet (i steden for parentes). 
\usepackage{booktabs}                     		% For å få tilgang til finere linjer (til bruk i tabeller og slikt).
\usepackage[font=small,labelfont=bf]{caption}	% For justering av figurtekst og tabelltekst.
\usepackage{todonotes}


\makeatletter
\def\ps@pprintTitle{%
  \let\@oddhead\@empty
  \let\@evenhead\@empty
  \let\@oddfoot\@empty
  \let\@evenfoot\@oddfoot
}
\makeatother

\usepackage[utf8]{inputenc}		% gjør slik at vi kan skrive æ,ø,å mm.
\usepackage{parskip}
\usepackage{hyperref}
\usepackage[noabbrev,capitalize]{cleveref}
\usepackage{subcaption}
\usepackage{physics}

% Denne setter navnet på abstract til Sammendrag
\renewenvironment{abstract}{\global\setbox\absbox=\vbox\bgroup
\hsize=\textwidth\def\baselinestretch{1}%
\noindent\unskip\textbf{Abstract}
\par\medskip\noindent\unskip\ignorespaces}
{\egroup}


% Disse kommandoene kan gjøre det enklere for LaTeX å plassere figurer og tabeller der du ønsker.
\setcounter{totalnumber}{5}
\renewcommand{\textfraction}{0.05}
\renewcommand{\topfraction}{0.95}
\renewcommand{\bottomfraction}{0.95}
\renewcommand{\floatpagefraction}{0.35}
\newcommand{\e}{\mathrm{e}}
%%%%%%%%%%%%%%%%%%%%%%%%%%%%%%%%%%%%%%%%%%%%%%%%%%%%%%%%%%%%%%%%%%%%%%%%%
\begin{document}

\begin{frontmatter}
\title{TFY4235 - The World of Quantum Mechanics}

\author[fysikk]{Karl Kristian Ladegård Lockert}
\address[fysikk]{Institutt for fysikk, Norges Teknisk-Naturvitenskapelige Universitet, N-7491 Trondheim, Norway.}
\begin{abstract}
When implementing different methods for solving the Schrödinger equation, we investigate the results and possible errors for three different systems, starting with a particle in a box. The computed energy eigenvalues are in good agreement with the analytic results for lower energies.  Introducing a potential barrier to the box, a tunneling phenomena occurs when preparing an initial 
state out of (almost) degenerate states and propagating over a suitable time span. 
Propagating the same initial state with the Crank-Nicolson scheme gives zero tunneling in the same time interval as in the plane-wave basis. The Euler-scheme is found unfit, attributed to the requirement of $H$ being unitary, which is not fulfilled. A relationship between the number of steps we are able to calculate (in the Euler scheme) and the discretization size, both spatial and temporal, is established. Lastly, a two-level system is studied. A numerical implementation of the time-evolution is presented,and used to compute the transition probabilities between two states. This is a result in good agreement with analytic approximations, with the added decoherence lowering the probability amplitude over time. 

\end{abstract}
\end{frontmatter}

\section{Introduction}

\begin{equation}\label{eq:SE}
i\hbar\pdv{\psi}{t} = H\psi
\end{equation}

\begin{equation} 
\label{eq:TISE}
H\psi_n = E_n\psi_n 
\end{equation}

Throughout the report, quantities with a bar ($\bar A$) that may have an analytic counterpart ($A$),  will represent quantities which are (to be) computed.
\section{Quantum mechanics in a box}

The Schrödinger equation tells us the time-evolution of the wave function, which, according to the Copenhagen interpretation of quantum mechanics, has the physical interpretation of a probability amplitude when squaring the absolute value. In our current setup, where we have
\begin{equation} 
V = \begin{cases}
0,\quad 0<x<L \\
\infty,\quad \text{otherwise}
\end{cases},
\end{equation}
there is 0\% chance of finding the particle inside the ``walls'' (at $x<0$ or $x> L$). Thus, the wave function must go to zero at these points. Defining $x' = x/L$ and $t'=t/t_0$, and inserting in \cref{eq:SE}.
\begin{align*} 
i\hbar\pdv{\psi}{t} &= i\hbar\pdv{\psi}{t'}\pdv{t'}{t}= \frac{-\hbar^2}{2mL^2}\pdv[2]{\psi}{{x'}}\\
\implies i\pdv{\psi}{t'} &= \frac{-\hbar}{2mL^2 \pdv{t'}{t}}\pdv{\psi}{x'^2}
\end{align*}
Thus, setting
\begin{equation}
\label{eq:normalized_stuff}
t' = \frac{\hbar}{2mL^2}t,\quad x' = \frac{x}{L}
\end{equation}
gives the wanted dimensionless equation
\begin{equation}
\label{eq:SE_dimensionless}
i\pdv{\psi}{t'} = -\pdv[2]{\psi}{{x'}}.
\end{equation}
Inserting our new variables in \cref{eq:TISE} with \[H = \frac{\hat p^2}{2m} + V(x) =  -\frac{-\hbar}{2mL^2} \pdv[2]{{x'}} + \tilde{V}(x'),\] we get
\begin{align} 
\frac{-\hbar^2}{2mL^2}\pdv[2]{\psi_n}{{x'}} &= E_n\psi_n \implies \nonumber\\
\label{eq:eig}
-\pdv[2]{\psi_n}{{x'}} &= \lambda_n\psi_n,
\end{align}
with the relation 
\begin{equation} 
\label{eq:energy_conversion}
\lambda_n = \frac{2mL^2 E_n}{\hbar^2}
\end{equation}
between the energy levels and the dimensionless eigenvalues. The boundary conditions that the wave function disappears in the walls, but the walls are now at $x' = 0$ and $x' = 1$. It is now clear that choosing $x_0 = L$ is suitable, as it makes us able to work on the simple domain $[0,1]$. Any other proportionality constant $\alpha\in \mathbb{R}$ such that $x' = \alpha x/ L$ should work as well, scaling both the time and energies by a factor $\alpha^2 $, as long as $x'$ is dimensionless. Other scaling possibilities are also possible, and have consequences for the analytic expressions for $\psi$. For example scaling the interval to be mirrored about $x = 0$ would only pick out the even terms in a Fourier series expansion.

Solving \cref{eq:eig} with the imposed boundary conditions can be done analytically in the following manner.
Since we are looking for solutions that have self-similar second derivatives with an extra minussign, we guess a solution on the form 
$\psi = A_n\sin(\sqrt{\lambda_n}x') + B_n\cos(\sqrt{\lambda_n}x')$. The boundary condition $\psi(x'=0) = 0$ gives $B_n = 0$, while the boundary condition $\psi(x'=1) = 0$ gives the restriction to $\lambda_n$ that $\sqrt{\lambda_n} = n\pi$, hence the labels $n$ are also justified. Our analytic solution is therefore 
\begin{equation}
\label{eq:eigenfunc}
\psi_n(x') = \mathcal{N}\sin(\pi n x'),
\end{equation} 
where $\mathcal{N}$ is a normalization constant to be decided.
\begin{align*} 
1 &= \braket{\psi_n}{\psi_n} = \mathcal{N}^2\int_{0}^{1}\dd{x'}\sin[2](n\pi x') \\
&=\mathcal{N}^2\int_0^1\dd{x'}\frac{1-\cos(2\pi n x')}{2}
= \frac{\mathcal{N}^2}{2} \\
&\implies \mathcal{N} = \sqrt{2},
\end{align*}
and we have the analytic solution as announced in \cite{assignment}.

The implementation of a finite-difference-scheme is done using sparse matrix formatting for the cases when the discretization step is very small. Otherwise a dense matrix format is sufficient for solving eigenvalues, using the ``eigsh''-function from NumPy's linear algebra library.

\begin{figure}
	\centering
	\begin{subfigure}{0.5\textwidth}
	\centering
	\includegraphics[width=\linewidth]{img/e_vs_lambda_N=5000}
	\caption{Computed eigenvalues against the analytic expression.}
	\end{subfigure}
	\hfill
	\begin{subfigure}{0.5\textwidth}
	\centering
	\includegraphics[width=\linewidth]{img/e_vs_n_N=5000}
	\caption{Computed eigenvalues against $n$.}
	\end{subfigure}
	\caption{Comparison of computed eigenvalues and the analytic expressions. }
	\label{fig:energy_comparison}
\end{figure}
In \cref{fig:energy_comparison} a comparison of the calculated eigenvalues against both the analytic expression for $\lambda_n$ and $n$ is shown for the case where the interval is discretized in 5000 points.  Notice that $\bar\lambda_n / \lambda_n$ is close to 1 for $n$ up to about 2000, where the energies are $2000^2$ times higher than the ground state.


To compute the error of a numerical solution, we must have a metric of some sort. Let us introduce
\begin{equation} 
E[\bar\psi_n(x')] = \int\dd{x'}||\bar \psi_n(x')|-|\psi_n(x')||^2
\end{equation}
as an example, where the bar denotes the numerical approximation to the analytic function.
Then, a comparison of the eigenvectors with respect to the number of discretization steps can be made. 
\begin{figure}
	\includegraphics[width=\linewidth]{img/discretization_step.png}
	\label{fig:discretization}
	\caption{Error of the ground state, first- and second excited states as a function of the discretization steps.}
\end{figure}
In \cref{fig:discretization}, this is shown for the three lowest energy levels. 


The implementation of 
\begin{equation} 
\alpha_n = \braket{\psi_n}{\Psi_0} = \int\dd{x'}\psi_n^*(x')\Psi_0(x')
\end{equation}
can be done in a simple fashion by taking optimized inner-product implementations for general vectors, so calculating the actual integral is actually not needed. 
\begin{figure}
	\centering
	\includegraphics[width=\linewidth, trim={2.5cm 0 0 0}, clip]{img/ortho.png}
	\caption{Logarithm (base 10) of the inner product of the 100 lowest excited eigenfunctions.}
	\label{fig:orthogonality}
\end{figure}
In \cref{fig:orthogonality}, the inner product of the first 100 eigenfunctions are shown, and indicates orthogonality of the states. The off-diagonals are $\sim 15 $ orders of magnitude smaller than the diagonal, which we might attribute to numerical artifacts or the imperfectness of the solutions, which we have seen is present, c.f. \cref{fig:energy_comparison,fig:discretization}. 

Using the reduced units previously discussed, we can express the full state
\begin{equation} 
\Psi(x, t) = \sum_n\alpha_n\exp(-\frac{iE_n t}{\hbar})\psi_n(x)
\end{equation}
as
\begin{align} 
\sum_n\alpha_n\exp(-\frac{iE_n t}{\hbar})&\psi_n(x) =\sum_n\alpha_n\exp(-\frac{iE_n t}{\hbar})\psi_n(x)\nonumber \\[1.5ex]\nonumber =\sum_n&\exp(-\frac{i\frac{\hbar^2\lambda_n}{2mL^2}\frac{2mL^2t'}{\hbar}}{\hbar})\psi_n(x') \\[1.5ex]\implies \Psi(x', t')&= \sum_n\alpha_n\exp(-i\lambda_n t')\psi_n(x').\label{eq:full_state}
\end{align}
Now using the ground state
\begin{equation} 
	\label{eq:init}
\Psi_0 = \sqrt2\sin(\pi x)
\end{equation}
as initial condition, we can compute the full state. 
\begin{figure}
	\centering
	\includegraphics[width=\linewidth]{img/init_state}
	\caption{Analytic and computed probability density of initial state $\Psi_0$ using $N = 1000$ as the number of discretization points}
	\label{fig:init_state}
\end{figure}
In \cref{fig:init_state} the initial states are plotted both for the analytic expression and the computed value at $t = 0$.
The normalization is also well-defined, as seen in \cref{fig:normalization}.
\begin{figure}
	\centering
	\includegraphics[width=\linewidth]{img/normalization.png}
	\caption{Normalization of the computed wave function for the time interval $t'\in [0,100]$ width $N =10000$ and $\Psi_0(x') = \delta(x'-\frac12)$.}
	\label{fig:normalization}
\end{figure}
For $\Psi_0$ as in \cref{eq:init}, this is expected, since $\Psi_0$ is orthogonal to $\Psi_{i\ne0}$, and $\Psi_0 $ is properly normalized itself. For $\Psi_0 = \delta(x'-1/2)$, however, this is more complicated. 
In this case, $c_n = \int\dd{x'}\psi_n^*\delta(x'-1/2)$
gives $c_n = \psi_n$, which means that only half of the eigenfunctions in \cref{eq:eigenfunc} will contribute. Numerically, however, we can only estimate the value of the Dirac-delta with a finite number of plane waves, in contrast to its integral representation $\delta \sim \int\dd{k}\e^{ikx}$.



\section{Adding a barrier to the boc-potential}
Let us now consider a potential barrier in the box, modeled by the dimensionless potential $\nu(x') = t_0V_0/\hbar$, where $t_0 =\frac{2mL^2}{\hbar}$ as in \cref{eq:normalized_stuff}, and is given by
\begin{equation} 
\nu(x') = \begin{cases}
0,\quad &0<x'<\frac{1}{3} \\
\frac{2mL^2V_0}{\hbar^2},\quad &\frac{1}{3}<x'<\frac{2}{3} \\
0,\quad &\frac{2}{3}<x'<1 \\
\infty ,\quad &\text{otherwise}
\end{cases}
\end{equation}
\subsection{Root finding}

Following ref. \cite{assignment}, the eigenvalues of the analytic solutions are given by the roots of the function $f(\lambda)$, shown in \cref{fig:f}. 
\begin{figure}
	\centering
	\includegraphics[width=\linewidth]{img/f}
	\caption{The analytic expression for finding the eigenvalues. }
	\label{fig:f}
\end{figure}
This is in good agreement with the computed eigenvalues. By using SciPy's \cite{2020SciPy-NMeth} optimized root-finding algorithm with the computed eigenvalues as the initial guess, we find three distinct eigenvalues below $\nu_0 = 10^3$.
Using the implementation of the root finder, the roots of $f(\lambda)$ is found more precicely, and the first couple of roots are shown in \cref{tab:roots}. 
\begin{table}
	\centering
\begin{tabular}{|c|c|}
	\hline
Computed $\bar\lambda$ & Roots of $f(\lambda) $\\
	\hline 
	73.49662578 & 73.93560016 \\ 

	73.49861656 &  \\ 

	291.77230762 & 293.49231502 \\ 
 
	291.79660664 &  \\ 
 
	644.62162712 & 648.26437316 \\ 

	645.10279446 &  \\
	\hline
\end{tabular}
	\caption{A comparison of computed eigenvalues and computed roots of the analytic expression for $f(\lambda)$. The units are given by \cref{eq:energy_conversion}. }
	\label{tab:roots}
\end{table}
Investigating the roots of $f(\lambda)$ for different values of $\nu_0$, we find that the value of $\nu_0$ that separates having one and no states with $\lambda < \nu_0$ is for 
\begin{equation} 
\nu_0 = 22.10526(3).
\end{equation}
\section{Step-by-step time evolution}

For a time-dependent Hamiltonian, an expansion in stationary eigenstates does not work in the same simple fashion. This is because the states $\psi_i$ are eigenvectors of the \textit{instantaneous} Hamiltonian. For different times , these vectors are not any more eigenvectors of the full Hamiltonian. 
That being said, with a step-by-step implementation we should get rid of the apparent degeneracy of the eigenvalues. 

\subsection{Euler scheme}
We now implement the Euler scheme for evaluating its applicability for the quantum problem. An interesting phenomena occur, where the function abruptly breaks down after a number of time steps. The approach to finding the number of steps before this breakdown is by looking at the normalization of the wave function. As this suddenly diverges, we can use this as a check in a while-loop for comparing step sizes in both temporal and spatial direction. 
\begin{figure}
	\centering
	\includegraphics[width=\linewidth]{img/cfl.png}
	\caption{The number of steps taken before the Wave Function breaks down against the Courant-Friedrichs-Lewy (CFL)-number on a log-log scale.}
	\label{fig:cfl}
\end{figure}

In \cref{fig:cfl} the steps before the breakdown of the simulation for the Euler-Scheme is shown. This suggests that for the a successful simulation in this scheme, we need $\Delta t' \ll (\Delta x')^2$. At thee same time we also want $\Delta x' \ll 1$, so this would require a very heavy computation. A better approach is to use a different numerical scheme. 

 
\subsection{Crank-Nicolson}

The implementation of this scheme is done by computing the LU-decomposition of $\left(1+\frac{i}{2}\Delta t'\hat H\right)$ and solving a linear system $A\vb x= b$ for each time step. 
As previously mentioned, we should get rid of the degeneracy of states. With this in mind, the phase  $\sim \frac{1}{\lambda_2-\lambda_1}$ should diverge, and we have to propagate the system infinitely in time.  By preparing the same state as in \cref{eq:psi0}, we can use the Crank-Nicolson scheme to propagate the function and check this. 
\begin{figure}
	\centering
	\includegraphics[width=\linewidth]{img/crank.png}
	\caption{Initial state and time-evolved state using the Crank-Nicolson scheme. }
	\label{fig:crank}
\end{figure}
As viewed in \cref{fig:crank}, there is now no tunneling, as opposed to the case when we project the states on a plane-wave basis, c.f. \cref{fig:tunneling}.

\subsection{Two level system}

\cref{fig:e1e2} shows the two lowest lying energy eigenvalues for $V(x, t)$ as introduced in  ref.\cite{assignment}. For $\nu_r = 0$, the energy difference is $\varepsilon_0 \simeq 5.6962$. 
\begin{figure}
	\centering
	\includegraphics[width=\linewidth]{img/e1_e2.png}
	\caption{The two lowest eigenvalues plotted against the varying potential $\nu_r$. }
	\label{fig:e1e2}
\end{figure}

The expectation value 
\begin{equation} 
\tau = \mel{g_0}{\hat H}{e_0}
\end{equation}
can be computed as the previous inner products, applying first the action of $\hat H$ to $\ket{e_0}$ and then taking the inner product. $\tau$ is found to take a linear shape 
\begin{equation}
	\tau(\nu_r) \simeq -0.429\nu_r.
\end{equation}

Let us next discretize the Volterra integral equation
\begin{equation} 
\ket{\psi(t=k\Delta t)}_I = \ket{\psi(0)} -\frac{i}{\hbar}\int_0^t\dd{t'}H_{1I}(t')\ket{\psi(t')}_I
\end{equation}
using the Trapezoidal rule
\begin{equation} 
\int_0^tf(t)\dd{t} \simeq \sum_{k=1}^{N}\frac{f((k-1)\Delta t)+f(k\Delta t)}{2}\Delta t.
\end{equation}
We get
\begin{align*} 
\ket{0} = &\ket{0} \\
\ket{\Delta t} = &\ket{0} - \frac{i}{\hbar}\Delta t H(0)\ket{0} \\
\ket{2\Delta t} = &\ket{0} - \frac{i}{\hbar}\Delta t \left(H(0)\ket{0} + H(\Delta t)\ket{\Delta t}\right) \\
&\vdots \\
\ket{n\Delta t} = &\ket{0}-\frac{i}{\hbar}\Delta t\left(\sum_{k=1}^{n-2}H(k\Delta t)\right. \\ 
&\qquad+ \left.\frac{H(0)\ket{0} + H((n-1)\Delta t)\ket{(n-1)\Delta t}}{2}\vphantom{(\sum_{k=1}^{n-2}}\right).
\end{align*}
The implementation of this discretization could be done by clever matrix manipulations. It could also be done with a simpler implementation, looping over all states previously calculated. This is a more ``brute-force'' way of doing it, but works.
\begin{figure}
	\centering
	\includegraphics[width=\linewidth]{img/transition_prob.png}
	\caption{The probability of finding the system in the first excited state, $\ket{e_0}$.}
	\label{fig:trans}
\end{figure}
In \cref{fig:trans} a comparison of the implementation and an approximate solution for the probability of finding $\ket{e_0}$ given that the system was initially in $\ket{\Psi(t=0)} = \ket{g_0}$. This was done for $\omega = \varepsilon_0$ and $\tau = 0.02\varepsilon_0$. Notice the decoherence that occur in the computed probability, not present in the analytic expression\cite{assignment}
\begin{equation} 
p(t) = \sin[2](\frac{t\tau}{2\hbar}).
\end{equation}
If we now vary $\omega$ and $\tau$, we may notice the following. Setting $\omega \ne \varepsilon_0$ will lower the probability of finding the system in the state $\ket{e_0}$, and shifting the peaks. Increasing $\tau$ does not seem to alter the amplitude of the probability, but shifts the peaks to earlier times. This is visualized in \cref{fig:probs}, where secondary oscillations are more pronounced than in \cref{fig:trans}. 
\begin{figure}
	\centering
	\includegraphics[width=\linewidth]{img/stuff.pdf}
	\caption{The transition probability $|\braket{g_0}{\psi(t)}|^2$ of three different configurations. }
	\label{fig:probs}
\end{figure}
If we choose $\tau$ too large, the method breaks down. This is clear from an interpretation of $\tau$ as a perturbation smallness parameter, in which a perturbation series would diverge for  $\tau>1$. 



\bibliographystyle{unsrt}
\bibliography{bib}

\end{document}
