\section{Introduction}
This assignment on ``The World of Quantum Mechanics'' is about different methods for computing properties of different quantum systems. 
We will investigate numerical solutions of different one-dimensional quantum mechanical systems,  all described by the Schrödinger equation
\begin{equation}\label{eq:SE}
i\hbar\pdv{\psi}{t} = H\psi.
\end{equation}
Starting with a particle in a box, we succsessfully express the full wave function as a linear combination of \textit{instantaneous} eigenstates. These are solutions of the time-independent Schrödinger equation
\begin{equation} 
\label{eq:TISE}
H\psi_n = E_n\psi_n.
\end{equation}
Introducing a potential barrier to the box, a tunneling phenomena occurs when preparing an initial state out of (almost) degenerate states,  and propagating this state over a suitable time span. 
For numerical evolution in time, two schemes are implemented. Both Euler, and Crank-Nicolson are able to reproduce the results obtained in the expansion of eigenstates, but the Euler-scheme requires a far finer discretization than what is computationally feasible. 
For a third system, presented with a time-dependent Hamiltonian, we discretize the Volterra integral equation for the formal solution of \cref{eq:SE}.
 
All of the implementations are written in Python 3.7, with heavy utilization of optimized libraries. 
These libraries include SciPy for linear algebra, NumPy for array manipulations and vectorization, dask for parallellization of multiple solvers at once, and Numba for ``just-in-time''-compilation of functions. 
 
The treatment of the problems will be done in reduces units. 
Throughout the report, quantities with a bar ($\bar A$) that may have an analytic counterpart ($A$),  will represent quantities which are (to be) computed where it is not clear. 